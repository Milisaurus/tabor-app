% Autor: Milan Vrbas <xvrbas01>
%        Jan Juračka <xrujac07>

\documentclass[a4paper, 12pt]{article} % definice třídy dokumentu a nastavení vlastností

% Packages
\usepackage[utf8]{inputenc} % nastavení kódování
\usepackage{graphicx} % obrázky
\usepackage{times} % font
\usepackage[czech]{babel} % čeština
\usepackage[T1]{fontenc} % nastavení kódování fontů
\usepackage[left=2cm, text={17cm, 24cm}, top=3cm]{geometry} % nastavení rozměrů stránky
\usepackage[unicode]{hyperref} % odkazy
\usepackage{fancyhdr} % zkrášlení stránky

\pagestyle{fancy}
\fancyhf{}
\fancyfoot[C]{\thepage} % čára na začátku stránky

\begin{document}
\begin{titlepage}
    \begin{center}
        \scalebox{0.15}{\includegraphics{../pictures/FIT_logo.png}} \\
        \vspace{\stretch{0.382}}
        \Huge{Závěrečný report} \\
        \Large{\textbf{Organizátor táborových bodů}} \\
        \large{ITU – Tvorba uživatelských rozhrání }
        \vspace{\stretch{0.618}}
    \end{center}

    {\large \today \hfill
        \large
        \begin{tabular}{l l}
        \textbf{Milan Vrbas} & \quad \textbf{xvrbas01}\\
        Jan Juračka           & \quad xjurac07      \\
        \end{tabular}
        }
\end{titlepage}

\tableofcontents
\thispagestyle{empty}
\newpage

\section{Zadání}
Úkolem je vytvořit webovou aplikaci, která bude sloužit vedoucím na letním táboře v 
Líseckém táboře na Vysočině k organizaci her a jejich bodování. Aplikace bude poskytovat 
přehledné zobrazení denních a celkových statistik pro jednotlivé týmy. \\
Aplikace umožní správu týmů, včetně jejich vytváření, mazání a úpravy. Každý tým bude mít 
přiřazený název, barvu, velitele a členy. Týmy budou sbírat body v táborových hrách, přičemž 
bodování bude podle tří předdefinovaných pravidel: Méně bodovaná, Více bodovaná a Velmi bodovaná. 
Uživatel bude mít také možnost vytvářet vlastní pravidla bodování pro specifické hry. \\
Informace o jednotlivých hrách a jejich výsledcích si budou uživatelé moci zobrazit v 
historii aktivit. Všechna táborová data (informace o týmech, jednotlivcích, 
hrách, výsledcích, historii her a nastavení bodování) budou uložena ve formátu JSON na 
backendové straně aplikace, což umožní snadné sdílení těchto informací s ostatními vedoucími.

\subsection{Odchylky od původního zadání}

\section{Rozdělení práce v týmu}

\section{Připomínky z kontrolní prezentace}

Během kontrolní prezentace jsme obdrželi užitečnou připomínku, kterou jsme následně 
zapracovali do našeho projektu. Tato připomínka se týkala zlepšení způsobu vytváření datového 
JSON souboru při zakládání nového tábora. 

Původní implementace vytvářela JSON soubor až ve chvíli, kdy uživatel dokončil vyplňování 
formulářů jednotlivých týmů a zvolil název tábora. Tento přístup však nebyl optimální, jak 
zdůraznil pan docent Vítězslav Beran, na kontrolní prezentaci. Navrhl, aby byl 
inicializační JSON soubor vytvořen ihned při zahájení procesu zakládání nového tábora. 
Tento soubor by obsahoval základní informace, jako je název tábora a prázdnou strukturu 
pro týmy, a následně by se doplnil o zbylá data po vyplnění a potvrzení formulářů.

Tento návrh jsme zapracovali následujícím způsobem:
- Při kliknutí na tlačítko *"Vytvořit nový"* se nyní okamžitě vytvoří
JSON soubor, který obsahuje inicializační strukturu dat a po vyplnění formulářů týmů a 
potvrzení vytvoření tábora se tato data zapíší do již existujícího JSON souboru.

Tato změna výrazně zjednodušila proces zpracování dat na backendu a umožnila nám lépe oddělit fázi inicializace od fáze úpravy dat. Díky tomu je nyní možné například uložit proces vytváření tábora, aniž by uživatel musel vyplnit všechny formuláře najednou.

Tento přístup rovněž přispívá k větší modularitě aplikace a usnadňuje práci s daty v dalších částech projektu, například při načítání nebo editaci tábora.


\section{Výsledky testování}

\end{document}